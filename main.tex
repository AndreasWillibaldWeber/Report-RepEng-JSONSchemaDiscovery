
% VLDB template version of 2020-08-03 enhances the ACM template, version 1.7.0:
% https://www.acm.org/publications/proceedings-template
% The ACM Latex guide provides further information about the ACM template

\documentclass[sigconf, nonacm]{acmart}

%% The following content must be adapted for the final version
% paper-specific
\newcommand\vldbdoi{XX.XX/XXX.XX}
\newcommand\vldbpages{XXX-XXX}
% issue-specific
\newcommand\vldbvolume{08}
\newcommand\vldbissue{01}
\newcommand\vldbyear{2023}
% should be fine as it is
\newcommand\vldbauthors{\authors}
\newcommand\vldbtitle{\shorttitle} 
% leave empty if no availability URL should be set
\newcommand\vldbavailabilityurl{https://github.com/AndreasWillibaldWeber/RepEng-JSONSchemaDiscovery}
% whether page numbers should be shown or not, use 'plain' for review versions, 'empty' for camera-ready
\newcommand\vldbpagestyle{plain} 

%Custom Commands
\newcommand{\red}[1]{\textcolor{red}{#1}}

\begin{document}
\title{RepEng Project: JSONSchemaDiscovery}

%%
%% The "author" command and its associated commands define the authors and their affiliations.
\author{Andreas W. Weber}
\affiliation{%
  \institution{University of Passau}
  \streetaddress{Innstr. 41}
  \city{Passau}
  \state{Germany}
  \postcode{94469}
}
\email{weber236@ads.uni-passau.de}

\maketitle

%%% VLDB block start %%%
\pagestyle{\vldbpagestyle}

%%% do not modify the following VLDB block %%
%%% VLDB block start %%%
\ifdefempty{\vldbavailabilityurl}{}{
\vspace{.3cm}
\begingroup\small\noindent\raggedright\textbf{PVLDB Artifact Availability:}\\
The source code, data, and/or other artifacts have been made available at \url{\vldbavailabilityurl}.
\endgroup
}
%%% VLDB block end %%%

\section{Introduction}

In science, the significance of reproducibility engineering grows as it becomes increasingly important to enhance the verifiability and adaptability of research findings by fellow scientists in many research areas. This report aims to construct such a reproducible package for the work of Angelo Augusto Frozza et al. \cite{8424731}, which presents an algorithm for extracting JSON schema and measuring the performance of the reference implementation \cite{JSONSchemaDiscovery2013}. Therefore, it describes the problems during the replication procedure, evaluates the findings against the original ones, and explains why only a reproduction package is intended. Furthermore, our experiment expects results similar to or better than the original findings.

\section{About the Original Work}

The \textit{JSONSchemaDiscovery} project focuses on deriving schemas from JSON or Extended JSON documents. The reference implementation of Sam Anzaroot et al. \cite{JSONSchemaDiscovery2013} loads these documents from the NoSQL database \textit{MongoDB}, derives the schemas and stores the results in the database. It also provides a web interface developed within the \textit{Angular} Framework. The backend is developed in \textit{Node.js}. It provides an API for the front end and collects and processes the data from the database through several worker threads. The algorithm intends to derive efficient JSON schemas of large unorganized JSON documents by consolidating information to construct separate schemas for each unique structure and joining them to a comprehensive global schema \cite{8424731}. Experiments on data sets such as \textit{DBPedia} and \textit{Foursquare} revealed that the \textit{JSONSchemaDiscovery} is equivalent or superior to similar methodologies \cite{8424731}. These experiments assess the document mapping quality and the processing time evaluation for comparison with findings from related work. The datasets for the experiments are not directly available as artefacts, and only one experiment has a good enough methodology description to create a reproduction for it. The \textit{JSONSchemaDiscovery} project was published on GitHub in two repositories: the original under the \textit{Apache2}\footnote{\url{github.com/feekosta/JSONSchemaDiscovery}} license and the fork under the \textit{CRAPL v0 BETA 1}\footnote{\url{github.com/gbd-ufsc/JSONSchemaDiscovery}} license.

\section{About the Reproduction}

\textit{Docker} containers provide a reliable and persistent environment to build and execute the \textit{JSONSchemaDiscovery} project. An official \textit{MongoDB} image from \textit{DockerHub}\footnote{\url{hub.docker.com/_/mongo}} is used for the database, which is essential because the runtime environment of the project and the database influence the experiment's results. For this reason, we also run the reproduction experiment in a separate custom container. \textit{Docker Compose} builds and runs the \textit{Docker} environment with a single command. The experiment container holds scripts to load the data to \textit{MongoDB}, execute the experiment, and collect the data. The main problem is bringing the \textit{JSONSchemaDiscovery} project to a stable version to let it run in a \textit{Docker} environment. It depends on many outdated packages that must be replaced step by step. Another problem is the availability of all artefacts. After contacting the authors, only the \textit{foursquare} data set \cite{ccelikten2016modeling} was made available \footnote{\url{www.dropbox.com/sh/j0bxw52b6fj46pm/AACOu60zgbNG1nKhnseYZ8uHa?dl=0}}. Another main problem was the imprecise description of performed experiments, which allowed only one experiment to be reproduced. Another challenge is automatically collecting the experiment data through the project's API. Often, the request times out before the experiment is finished or the experiment is not complete. The program calling the API must be able to handle this behaviour.


\begin{table}[ht]
  \caption{Results obtained from \textit{Foursquare} datasets: Number of JSON documents (N-JSON), raw schemas (RS), raw schemas with ordered structure (ROrd), time to obtain raw schemas (TB), and the total time  (TT) per minute.}
  \label{tab:foursquare}
  \begin{tabular}{lrrrrrr}
    \toprule
    Collection & N-JSON &  RS & ROrd &   TB &    TT & TB/TT \\
    \midrule
    venues     &      2 & 257 & 117 &  7,47 &  7,52 & 99.33 \\
    checking   &     11 &   2 &   2 & 35,27 & 35,52 & 99.29 \\
    tweets     &     17 &  23 &  16 & 53,11 & 53,44 & 99.38 \\
    \bottomrule
\end{tabular}
\end{table}

\section{Conclusion}

Currently, the \textit{JSONSchemaDiscovery} project still needs to be entirely dockerized, which is the main problem of this work, and further evaluation is required to gather reliable results. Table 1 shows the original findings, which are attempted to be confirmed by a reproduction using the same artefacts and methods.

%\clearpage

\bibliographystyle{ACM-Reference-Format}
\bibliography{references}

\end{document}
\endinput
